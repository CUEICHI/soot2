%%%%%%%%%%%%%%%%%%%%%%%%%%%%%%%%%%%%%%%%%%%%%%%%%%%%%%%%%%%%%%%%%%%%%%%%%%%%%%
%%% タイトル,著者,所属,概要
% 日本語タイトル
\jtitle{
Sootography : 煤を使った描画システム% タイトル記述.適宜改行を入力.
}

% 英語タイトル
\etitle{
Sootgraphy: Drawing machine with soot of candle.
}

% 日本語著者
% 所属参照は好みに応じて \dagger などを用いてもよい.
\jauthor{
羽田久一\(^{1)}\){\small (正会員)}
%~~~ % 氏名の間隔はチルダ記号や hspace 等で適宜調節のこと.
%芸術科学次郎\(^{2)}\){\small (正会員)}
}

% 英語著者
\eauthor{
Hisakazu Hada\(^{1)}\)
%~~~
%Jiro Geijutsu-Kagaku\(^{2)}\)
}

% 日本語所属
\jaffiliation{
1) 東京工科大学メディア学部
~~~
%2) 芸術科学大学芸術科学部
}

% 英語所属
\eaffiliation{
1) Schoole of Media Science, Tokyo University of Technology
%2) Department of Art and Science, The University for Art and Science
}

% 連絡先電子メールアドレス(省略可)
% (スパム対策は著者自身の判断によって措置すること.
% このサンプルでは「@」を2バイト文字にすることで対応してある.)
\email{
hadahskz@edu.teu.ac.jp
}
