\section{はじめに}

画材としての煤は古代より広く利用されている.煤をそのまま利用する,あるいは
煤をあつめ黒い塗料として利用することで絵画を描画することは広く行われている.

蝋燭の煤を利用し,直接すすを画材に塗布することで描画する方法も広く行われており,
数々の作品が発表されている.

筆者らは
煤の持つ光を反射しない黒色の色
煤の付着する様子
蝋燭の炎の揺らぎ
などに着目し sootoid\cite{sootid}を作成した.
しかしながら,機構の制約として煤を塗布する太さに
自然のゆらぎ以上の変化をもたらせないこと,途中での消火・点火を行わないため
一筆書きとなるような画像しか描画できないことなどの課題が残った.

本研究ではこれらの課題に着目し,新たに煤生成の制御機構を構築することで
より表現力の豊かなさまざまな種類の描画を可能としたものである.

