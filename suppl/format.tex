\section{投稿論文の書式}

\subsection{ページ設定およびページ数}

論文本体のページ設定は,A4とする.このページ設定で不都合なコンテンツがある場合は,
静止画であっても論文本体に含めずに添付ファイルで提出していただきたい.

論文本体のページ数は特に規定しない.
ただし原則として,本文の文字数を以下の通り規定する.
\begin{itemize}
\item 本文が日本語2500文字または英語1500単語以内の論文は原則として
	ショートペーパー,それ以上の論文はフルペーパーとして扱う.
\item フルペーパーの場合,本文の長さを,日本語15000文字以内,
	英語9000単語以内,と規定する.
	それ以上の長さの論文を投稿したいときは,論文の一部を付録資料として,
	別ファイルにて提出されたものを受け付ける.
\end{itemize}

\subsection{論文の構成}

論文本体には,まず冒頭に以下の内容を記述すること.
本 LaTeX ファイルの冒頭部分を参照のこと.

\begin{itemize}
\item 論文題名(原則として,和文・英文の両方)
\item 著者名(原則として,和文・英文の両方)
\item 著者所属名(原則として,和文・英文の両方)
\item 著者連絡先は,論文本体には書いても書かなくてもよい
\item アブストラクト(原則として,和文・英文の両方)
\end{itemize}
続いて本文以降,以下の内容を記述のこと.
本文の書式は,原則として2段組とする.
\begin{itemize}
\item 本文(原則として,和文または英文)
\item 参考文献(本文と同一の言語で)
\item 図表(本文と同一の言語で)
\item 著者略歴(本文と同一の言語で)
\end{itemize}

なお当論文誌では,いわゆるダブルブラインドレビュー
(査読者に対して著者情報を伏せた形式での査読)を採用していない.
そのため,\textbf{査読原稿にあっても,著者名,著者所属名は省略しないこと.}


